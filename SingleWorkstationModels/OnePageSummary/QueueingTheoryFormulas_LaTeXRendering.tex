 \documentclass[11pt]{article}
\usepackage[left=1in, right=1in, top=1in, bottom=1in]{geometry}
\usepackage{amsmath, amsfonts, amssymb}
\usepackage{listings}
\usepackage{graphicx}
\usepackage{color}
%%%%%
\usepackage{xfrac}
\usepackage{nicefrac}


\begin{document}

\noindent  \textbf{M/M/1}  ~~$[WIP, WIP_q, CT, CT_q, TH, U] = f(\mu_p, \mu_{ia})$
\begin{align*}
U 		&= \frac{\mu_p}{\mu_{ia}}	\qquad	\text{(any stable single workstation)}\\
WIP 	&= \frac{U}{1-U}	\qquad	\text{(from Markovian assumptions)} \qquad \qquad \qquad \qquad \qquad \qquad \qquad \qquad\\
TH		&= \frac{1}{\mu_{ia}}	\qquad	\text{(any stable single workstation)}\\
CT		&= \frac{WIP}{TH}	\qquad	\text{(Little's Law)}\\
CT_q	&= CT - \mu_p 	\qquad	\text{(any single workstation)}\\
WIP_q	&= TH * CT_q	\qquad	\text{(Little's Law)}
\end{align*}

\noindent  \textbf{M/M/k}  ~~$[~] \approx f(\mu_p, \mu_{ia}, k)$
\begin{align*}
U 		&= \frac{\mu_p}{k \mu_{ia}}		\qquad	\text{(any stable single workstation)} \qquad \qquad \qquad \qquad \qquad \qquad \qquad \qquad \quad\\
CT_q	&\approx \frac{U^{\sqrt{2(k+1)}-1}}{k(1-U)} \mu_p 	\quad	\text{(Sakasegawa, 1977)}\\
		&\hdots\\
CT		&= CT_q + \mu_p 	\qquad	\text{(any single workstation)}\\
TH		&= \frac{1}{\mu_{ia}}	\qquad	\text{(any stable single workstation)}\\
WIP_q	&= TH * CT_q	\qquad	\text{(Little's Law)}\\
WIP	&= TH * CT	\qquad	\text{(Little's Law)}
\end{align*}

\noindent  \textbf{G/G/1}  ~~$[~] \approx f(\mu_p, \sigma^2_p, \mu_{ia}, \sigma^2_{ia})$ with $c_{ia} = \frac{\sigma_{ia}}{\mu_{ia}}$ and $c_p = \frac{\sigma_p}{\mu_p}$
\begin{align*}
U 		&= \frac{\mu_p}{\mu_{ia}}	\qquad	\text{(any stable single workstation)}\\
CT_q	&\approx \left( \frac{c_{ia}^2 + c_p^2}{2} \right)  \left( \frac{U}{1-U} \right)  \mu_p 	\quad	\text{(Kingman $\equiv$ VUT)} \qquad \qquad \qquad \qquad \qquad \qquad \qquad\\
		&\text{Continue as with M/M/k \ldots}
\end{align*}

\noindent  \textbf{G/G/k}  ~~$[~] \approx f(\mu_p, \sigma^2_p, \mu_{ia}, \sigma^2_{ia}, k)$
\begin{align*}
U 		&= \frac{\mu_p}{k \mu_{ia}}		\qquad	\text{(any stable single workstation)}\\
CT_q	&\approx \left( \frac{c_{ia}^2 + c_p^2}{2} \right)  \left( \frac{U^{\sqrt{2(k+1)}-1}}{k(1-U)} \right)  \mu_p 	\quad	\text{(VUT, Sakasegawa)} \qquad \qquad \qquad \qquad \qquad \qquad \qquad \qquad\\
		&\text{Continue as with M/M/k \ldots}
\end{align*}

%\noindent  \textbf{M/M/k} and \textbf{G/G/1}, \textbf{G/G/k}
%\begin{align*}
%CT		&= CT_q + \mu_p 	\qquad	\text{(any single workstation)}\\
%TH		&= \frac{1}{\mu_{ia}}	\qquad	\text{(any stable single workstation)}\\
%WIP_q	&= TH * CT_q	\qquad	\text{(Little's Law)}\\
%WIP	&= TH * CT	\qquad	\text{(Little's Law)}
%\end{align*}


%%%%%
\newpage

\noindent  If \textbf{Preemptive Outages}\\
\noindent  - Time between outages has average $M_{TTF}$, $CV \approx 1$\\
\noindent  - Outages have mean and variance $(M_{TTR}, \sigma_{TTR}^2)$\\
\noindent  Then correct $(\mu_p, \sigma^2_p)$ to effective $(\mu_e, \sigma^2_e)$:
\begin{align*}
A 				&= \frac{M_{TTF}}{M_{TTF} + M_{TTR}}\\
\mu_e 			&= \frac{\mu_p}{A}\\
\sigma^2_e 	&= \left(\frac{\sigma_p}{A}\right)^2 + \frac{\left(M_{TTR} + \sigma_{TTR}^2\right)(1-A)\mu_p}{A \cdot M_{TTR}} \qquad \qquad \qquad \qquad \qquad \qquad \qquad \qquad \qquad
\end{align*}

\noindent  If \textbf{Nonpreemptive Outages}\\
\noindent  - Number of jobs between outages has average $N$, $CV \approx 1$\\
\noindent  - Outages have mean and variance $(\mu_s, \sigma_s^2)$\\
\noindent  Then correct $(\mu_p, \sigma^2_p)$ to effective $(\mu_e, \sigma^2_e)$:
\begin{align*}
\mu_e 			&= \mu_p + \frac{\mu_s}{N}\\
\sigma^2_e 	&= \sigma_p^2 + \frac{\sigma_s^2}{N} + \frac{N-1}{N^2} \mu_s^2 \qquad \qquad \qquad \qquad \qquad \qquad \qquad \qquad \qquad \qquad \qquad \qquad
\end{align*}

\noindent  If both \textbf{Preemptive and Nonpreemptive Outages}\\
\noindent  Then apply the formulas consecutively.


%%%%%
%\vspace{10mm}
%\noindent  If a \textbf{Finite Buffer} for the single workstation's queue, \ldots


%%%%%
\newpage

\noindent  If \textbf{Process Batching, Parallel Processing}\\
\noindent  - G/G/1 queue with batch size $b$\\
\noindent  - Whole-batch processing time $\mu_p$
\begin{align*}
U 		&= \frac{\mu_p}{b \mu_{ia}}		\qquad	\text{(any stable single workstation)}\\
CT_q	&\approx \left( \frac{\frac{c_{ia}^2}{b} + c_p^2}{2} \right)  \left( \frac{U}{1-U} \right)  \mu_p 	\quad	\text{(Kingman)}\\
CT		&= \frac{(b-1)\mu_{ia}}{2}\mu_p + CT_q + \mu_p 	\quad \text{(add wait-to-batch)} \qquad \qquad \qquad \qquad \qquad \qquad \qquad \qquad\\
		&\hdots\\
TH		&= \frac{1}{\mu_{ia}}	\qquad	\text{(any stable single workstation)}\\
WIP_q	&= TH * CT_q	\qquad	\text{(Little's Law)}\\
WIP	&= TH * CT	\qquad	\text{(Little's Law)}
\end{align*}

\noindent  If \textbf{Process Batching, Serial Processing}\\
\noindent  - G/G/1 queue with batch size $b$\\
\noindent  - Per-element processing time $\mu_p$\\
\noindent  - Setup time between batches $s$\\
\noindent  - SCV for batch process \& setup time $c_e^2$\\
\noindent  - Assume $c_e^2=0.5$ and $c_{ia}^2=1$:
\begin{align*}
U 		&= \frac{\mu_p + \nicefrac{s}{b}}{\mu_{ia}}\\
CT_q	&\approx \left( \frac{c_{ia}^2 + c_p^2}{2} \right)  \left( \frac{U}{1-U} \right)  \mu_p 	\quad \text{(Kingman)}\\
WIBT	&= \frac{b-1}{2} \mu_p \text{ (lot splitting),} ~~ (b-1) \mu_p \text{ (no splitting)} \qquad \qquad \qquad \qquad \qquad \quad \qquad \qquad \qquad\\
CT		&= CT_q + s + WIBT + \mu_p\\
		&\text{Continue as with Parallel Processing \ldots}
\end{align*}


%%%%%
\newpage

\noindent  For \textbf{serially-arranged} workstations,\\
\noindent  the departure process from $W_1$ is the arrival process at $W_2$.\\
$(\mu_d, c_d^2) = f(c_{ia}^2, c_p^2, k, U, TH)$:
\begin{align*}
\mu_d 	&= \frac{1}{TH}\\
c_d^2 	&= 1 + (1-U^2)(c_{ia}^2-1) + \frac{U^2}{\sqrt{k}}(c_p^2-1) \qquad \qquad \qquad \qquad \qquad \qquad \qquad \qquad \qquad\\
		&\text{which for $k=1$ reduces to:} \\
c_d^2 	&= U^2 c_p^2 + (1-U^2)c_{ia}^2
\end{align*}

\vspace{3mm}

\noindent For a workstation with \textbf{parallel downstream branches},\\
\noindent If the branch's routing control logic is probabilistic\\
\noindent (prob $p_1$ of routing to $W_{2a}$, prob $p_2$ of routing to $W_{2b}$, \ldots)\\
\noindent and if $W_1$'s departure process has $(\mu_d, c_d^2 \approx 1)$\\
\noindent Then results for ``splitting'' a Poisson process apply.\\
\noindent - Arrivals at $W_{2a}$ will have $(p_1 \mu_d, c_d^2 \approx 1)$\\
\noindent - Arrivals at $W_{2b}$ will have $(p_2 \mu_d, c_d^2 \approx 1)$, etc.

\end{document}